\chapter{Adaptiver Gitteralgorithmus}
Mit berechenbaren Fehlerindikatoren die Verlässlichkeits- und Effizientsabschätzungen erfüllen, folgt ein Algorithmus zur effizienteren Approximation der Differentialgleichung nach dem Schema Lösen-Approximieren-Markieren-Verfeinern. Dabei entsteht eine Folge von Triangulierungen, indem von Glied zu Glied die Elemente mit den größten Indikatoren verfeinert werden.
\begin{algorithmus}[Adaptives Gitter]
	Wähle $\mathscr{T}_0, \theta\in (0,1],\varepsilon_{stop}>0$. Setze $k=0$.
	\begin{itemize}
		\item[(1)] berechne Approximation $u_k\in\mathscr{S}_0^1(\mathscr{T}_k)$
		\item[(2)] berechne $\eta_T(u_k)$ für alle $T\in\mathscr{T}_k$
		\item[(3)] wenn $\eta_{\mathscr{R}}(u_k) \leq \varepsilon$ dann stopp
		\item[(4)] wähle Menge $M_k\subset\mathscr{T}_k$ sodass
		\[ \sum_{T\in M_k} \eta_T^2(u_k) \geq \theta\sum_{T\in\mathscr{T}_k}\eta_T^2(u_k)\]
		\item[(5)] verfeinere jedes $T\in M_k$ und anliegende Elemente, um so neue Triangulierung $\mathscr{T}_{k+1}$ zu erhalten
		\item[(6)] $k \rightarrow k + 1$ und weiter mit (1)
	\end{itemize}
\end{algorithmus}
Im Schritt (5) müssen nach dem Verfeinern der Elemente in $M_k$ noch weitere Elemente verfeinert werden, um sicher zu stellen, dass eine konforme Triangulierung ohne hängende Knotenpunkte entsteht. Dafür können verschiedene Strategien verwendet werden.

\section{Rot-Grün-Blau Verfeinerung}
Da bei der Rot-Grün-Blau Verfeinerung immer die längste Seite verkürzt wird, ist sicher gestellt das die Dreiecke nicht degenerieren. Der Name Rot-Grün-Blau kommt von der Bezeichnung der Aufteilungen bei der drei, eine oder zwei Seiten halbiert werden (Abbildung \ref{dreieck})
\newpage
\begin{algorithmus}[Rot-Grün-Blau Verfeinerung]
    Sei $M_k\subset \mathscr{T}_k$ Menge der zu verfeinernden Elemente.
	\begin{itemize}
		\item[(1)] Markiere alle Seiten $S \in \mathscr{S}_k$ mit $S \subset \p T, T\in M_k$
		\item[(2)] Markiere weitere Seiten, sodass für alle markierten Seiten $S\subset \p T$  auch die längste Seite von T
		\item[(3)] Verfeinere Elemente je nach Anzahl markierter Seiten mit  Rot-,Blau- oder Grünverfeinerung für drei, zwei oder einer Seite markiert.
	\end{itemize}
\end{algorithmus} 

\begin{figure}[!htbp]
	\begin{center}
		\includegraphics[width=16cm]{pics/redref.png}
	\end{center}
	\caption{\label{dreieck}Rote, grüne und blaue Verfeinerung eines Dreiecks. Die unter Kante entspricht der längsten Kante}
\end{figure}

\begin{figure}[!htbp]
	\begin{center}
		\includegraphics[width=16cm]{pics/refin.png}
	\end{center}
	\caption{Verfeinerung von markierten Elementen einer Triangulierung und weiteren Elementen, um hängende Knoten zu verhindern}
\end{figure}

\section{Bisektionsverfahren}
Eine Alternative zur Rot-Grün-Blau Verfeinerung bietet die Bisektion.  Zu jedem Element wird ein Verfeinerungsknoten gespeichert. Beim Verfeinern eines Elements wird ein neuer Knoten in der Mitte der Seite gegenüber dieses Knotens erzeugt und mit dem Verfeinerungsknoten verbunden. Für die zwei so entstandenen Elemente ist der neu erzeugte Knoten der Verfeinerungsknoten (Abbildung \ref{schbisek}). Es werden dabei erst dann Elemente verfeinert, wenn die Verfeinerungsseite, also die Seite gegenüber des Verfeinerungsknotens, entweder auf dem Rand liegt oder Verfeinerungsseite des anliegenden Elements ist. So bleibt die Triangulierung in jedem Schritt regulär (Abbildung \ref{bisec}).
\newpage
\begin{algorithmus}[newest Vertex Bisektion]
	Sei $T\in \mathscr{T}_k$ mit Verfeinerungsseite $S$ zu verfeinerndes Element.
	\begin{itemize}
		\item[(1)] Wenn $S \subset \p \Omega$, dann erzeuge Knoten z als Mittelpunkt von S. Verbinde z mit dem Verfeinerungsknoten von T. Setze z als Verfeinerungsknoten der beiden so entstandenen Elemente. stopp  
		\item[(2)] Sei $D \in \mathscr{T}_k$ mit $D\cap T = S$. Wenn S nicht Verfeinerungsseite von D, dann wende newest Vertex Bisection auf D an.
		\item[(3)] Erzeuge Knoten z als Mittelpunkt von S. Verbinde z mit dem Verfeinerungsknoten von T und D. Setze z als Verfeinerungsknoten der vier so entstandenen Elemente.
	\end{itemize}
\end{algorithmus}
Nach Schritt 2 ist garantiert, dass S Verfeinerungsseite von D ist. \\

\begin{figure}[!htbp]
	\begin{center}
		\includegraphics[width=16cm]{pics/bisec1.png}
	\end{center}
	\caption{\label{schbisek}Schrittweise Verfeinerung eines Elements mithilfe von Bisektion}
\end{figure}

\begin{figure}[!htbp]
	\begin{center}
		\includegraphics[width=16cm]{pics/bisec2.png}
	\end{center}
	\caption{\label{bisec}Verfeinerung eines Elements (markiert) einer Triangulierung und weiteren Elementen, um hängende Knoten zu verhindern, mithilfe von newest Vertex Bisektion}
\end{figure}

Auch im dreidimensionalen findet das Bisektionsverfahren Anwendung. Im Gegensatz zum zweidimensionalen Analogon reicht ein Knoten jedoch nicht aus um eine eindeutige Teilung eines Elements zu bestimmen. Daher wird jedem Element eine Verfeinerungskante zugewiesen. Um ein Element zu teilen, wird nun ein Knoten am Mittelpunkt der Verfeinerungskante erzeugt. Die Ebene durch diesen neuen Knoten und die zwei Knoten, die der Verfeinerungskante gegenüberliegen teilt den Tetraeder in zwei neue Tetraeder. Verfahren bei denen die neue Verfeinerungskante immer eine der beiden zum neuen Knoten gegenüberliegenden Kanten ist, werden den newest Vertex Bisektionsverfahren zugeordnet wie zum Beispiel der Algorithmus von Kossaczky (Abbildung \ref{kos}).

\begin{figure}[!htbp]
\begin{center}
	\includegraphics[width=16cm]{pics/bisec3.png}
\end{center}
\caption{\label{kos}Verfeinerung eines Tetraeders beginnend mit sechs Tetraedern mithilfe des Algorithmus von Kossaczky}
\end{figure}