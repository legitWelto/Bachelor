\chapter{Beispielkapitel}
Zu Beginn eines Kapitels kann eine kurze Zusammenfassung des
Inhalts des Kapitels stehen, die nicht mehr als zehn Zeilen
umfassen sollte. 

\section{Lokale Ungleichungen}
\begin{definition}
	Sei $z \in \mathcal{N}_h$ Knoten. Definiere $\omega_z \subset \Omega$ als
	\[
	\omega_z = supp(\varphi_z)
	\]
	Und $h_z = diam(\omega_z)$ Durchmesser von $\omega_z$
\end{definition}

\begin{lemma}[Lokale Poincaré Ungleichung]
	Sei $v \in H^1_D(\Omega)$ un $z\in \mathcal{N}_h.$ Sei $v_z= 0$ für $z\in \Gamma_D$ und 
	\[ v_z = |\omega_z|^{-1} \int_{\omega_z} v \quad dx
	\]
	sonst. Dann gibt es für alle $h > 0$ und $z \in \mathcal{N}_h$ eine Konstante $c_{p,z}>0$, sodass:
	\[
	\|v-v_z\|_{L^2(\omega_z)}\leq c_{p,z}h_z\|\nabla v\|_{L^2(\omega_z)}.
	\]
\end{lemma}

\begin{lemma}[Lokale Spur Ungleichung]
	Sei $S \in  \mathcal{S}_h$ und $T_S \in \mathscr{T}_h so, dass S \subset \p T_s$. Dann
	gibt es die Konstante $c_{Tr} > 0$, sodass für alle $v\in H^1(\Omega)$ gilt:
	\[
	\|v\|^2_{L^2(S)} \leq c^2_{Tr}(h_S^{-1}\|v\|^2_{L^2(T_S)}+h_S\|\nabla v\|^2_{L^2(T_S)}).
	\]
\end{lemma}
\section{Clément Quasi Interpolant}
\begin{definition}
    Sei $v \in L^1(\Omega) , z \in \mathcal{N}_h$
	\[
	v_z =  \left\{
	\begin{array}{ll}
		|\omega_z|^{-1} \int_{\omega_z} v dx& \text{f\"ur } z \in \mathcal{N}_h \setminus \Gamma_D\\
		0 &  \text{f\"ur } z \in \mathcal{N}_h \cap \Gamma_D
	\end{array}\right.
	\]
	Definiere den Clément Quasi Interpolant$\mathscr{J}_hv \in \mathscr{S}_D^1(\mathscr{T}_h)$ von v als
	\begin{displaymath}
		\mathscr{J}_hv = \sum_{z\in\mathcal{N}_h} v_z \varphi_z
	\end{displaymath}
\end{definition}
\begin{theorem}(Quasi Interpolant Abschätzung)
	Es existiert $c_\mathscr{J} > 0$, so dass f\"ur alle $v\in H^1_D(\Omega)$ gilt:
	\[
	\|\nabla\mathscr{J}_hv\| +\|h^{-1}_\mathscr{T}(v-\mathscr{J}_hv)\| +\|h^{-\frac{1}{2}}_\mathscr{S}(v-\mathscr{J}_hv)\|_{L^2(\cup\mathscr{S}_h)} \leq c_\mathscr{J}\|\nabla v\|
	\]
	mit $h_\mathscr{T}: \Omega \rightarrow  \R \text{ und } h_\mathscr{S}: \cup \mathscr{S}_h \rightarrow  \R$, definiert durch $h_\mathscr{T}|_T = h_T \text{ und }  h_\mathscr{S}|_S = h_S$ \\ f\"ur alle $T\in\mathscr{T}_h, S\in\mathscr{S}_h$ 
\end{theorem}
\section{A Posteroiori Fehlerabschätzung}
\begin{definition}[Sprung]
	Sei $u_h\in\mathscr{S}^1(\mathscr{T}_h)$ und $S\in\mathscr{S}_h$ mit $S=T_1 \cap T_2$ für $T_1,T_2 \in \mathscr{T}_h$ verschieden, mit äußeren Normalen $n_{T_1,S},n_{T_2,S}$ auf S. Definiere den Sprung von $\nabla u_h$ über S als
	\[
	\llbracket \nabla u_h \cdot n_S\rrbracket = \nabla u_h|_{T_1} \cdot n_{T_1,S} + \nabla u_h|_{T_2} \cdot n_{T_2,S}
	\]
	Setze $\llbracket \nabla u_h \cdot n_S\rrbracket = 0$ f\"ur $S\subset \Gamma_D$
\end{definition}
\begin{definition}[Verfeinerungsindikator]
	Für $u_h \in \mathscr{S}^1(\mathscr{T}_h) \text{ und } T \in \mathscr{T}_h$, definiere Verfeinerungsindikator $\eta_T(u_h)$ durch
	\[
	\eta^2_T(u_h) = h^2_T\|f\|^2_{L^2(T)} + \sum_{S\in\mathscr{S}_h,S\subset\p T} h_S\|\llbracket \nabla u_h \cdot n_S\rrbracket\|^2_{L^2(S)} \]
	und den Fehlerschätzer $\eta_\mathscr{R}$ durch
	\[
	\eta_\mathscr{R}^2(u_h)=\sum_{T\in\mathscr{T}_h} \eta^2_T(u_h)
	\]
\end{definition}
\begin{theorem}[A Posteriori Fehlerabschätzung]
	Sei $u\in H^1_0(\Omega)$ schwache Lösung des Poisson-Problems \[-\Delta u = f ,\quad \Gamma_D = \p \Omega \text{ und } u|_{\p \Omega} = 0, \] und $u_h \in \mathscr{S}^1_0(\mathscr{T}_h)$ Galerkin-Approximation von u. Dann gibt es $c_R > 0$ , sodass:
	\[\|\nabla(u-u_h)\|\leq c_R\eta_\mathscr{R}(u_h)
	\]
\end{theorem}
\section{Effizienz}
\begin{definition}
	Für $S \in \mathscr{S}_h$ Seite, definiere 
	\[
		\omega_S= int(\bigcup_{T\in\mathscr{T}_h,S\subset \p T}T)
    \]
    und für ein Element $T \in \mathscr{T}_h$ definiere
    \[
    \omega_T = \bigcup_{z\in\mathscr{N}_h,z\in T} \omega_z
    \]
\end{definition}
\begin{lemma}
	\leavevmode
	\begin{itemize}
		\item[1)]
		Für alle $T\in\mathscr{T}_h$ mit $T =conv\{z_1,z_2,...,z_{d+1}\}$ gibt es Konstanten $c_{e,1},c_{e,2}$,  sodass für $b_T = \varphi_{z_1}\varphi_{z_2}\cdots\varphi_{z_{d+1}} \in H^1(\Omega)\cap C(\overline{\Omega})$ gilt:
		\[
		supp(b_T)\subset T,\quad \int_{T} b_T =c_{e,1} |T|, \quad \|\nabla b_T\|_{L^2(T)} \leq c_{e,2}h_T^{d/2-1}
		\]
		\item[2)]
		Für alle $S\in\mathscr{S}_h$ mit $S =conv\{z_1,z_2,...,z_{d}\}$ gibt es Konstanten $c_{s,1},c_{s,2}$,  sodass für $b_S = \varphi_{z_1}\varphi_{z_2}\cdots\varphi_{z_{d}} \in H^1(\Omega)\cap C(\overline{\Omega})$ gilt:
		\[
		supp(b_S)\subset \omega_S,\quad \int_{S} b_S =c_{s,1} |S|, \quad \|\nabla b_S\|_{L^2(\omega_S)} \leq c_{s,2}h_S^{d/2-1}
		\]
	\end{itemize}
\end{lemma}
\begin{theorem}[Lokale Effizienz]
	Sei $d=2, \Gamma_D = \p \Omega$ und $f\in L^2(\Omega)$ elementeweise konstant. Dann existiert eine Konstante $c_E > 0$, sodass für alle $T\in\mathscr{T}_h$ gilt
	\[
		c_E \eta_T^2(u_h)\leq \|\nabla (u-u_h)\|^2_{L^2(\omega_T)}
	\] 
\end{theorem}