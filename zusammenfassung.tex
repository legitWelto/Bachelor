\chapter*{Zusammenfassung}
\addcontentsline{toc}{chapter}{Zusammenfassung} 

Mithilfe des erfahrungsbasierten, berechenbaren Fehlerindikators
\[
\eta_\mathscr{R}(u_h)=\left(\sum_{T\in\mathscr{T}_h} \eta^2_T(u_h)\right)^{1/2}
\]
mit
$
\eta^2_T(u_h) = h^2_T\|f\|^2_{L^2(T)} + \sum_{S\subset\p T} h_S\|\llbracket \nabla u_h \cdot n_S\rrbracket\|^2_{L^2(S)} 
$, ist es möglich den Fehler einer Approximation des Poisson Problems zu kontrollieren, also den Fehler nach oben und unten zu beschränken. So kann die Approximation, insbesondere lokal, ohne Kenntnis der exakten Funktion bezüglich der Exaktheit beurteilt werden. Es ist bekannt, dass mit der Gitterweite der Fehler sinkt. So folgt für jedes Element $T \in \mathscr{T}_k$ und Elemente $T_i'\in \mathscr{T}_{k+1}$ mit $\overset{.}{\cup} T_i' =T$, da $\eta_T$ den Fehler nach oben beschränkt, dass $\eta_T\geq \sum \eta_{T'_i}$ Somit wird $\eta_\mathscr{R}$ kleiner, wenn einzelne $T$ verfeinert werden. Da $\eta_\mathscr{R}$ den Fehler nach oben beschränkt, reicht es also $\eta_\mathscr{R}$ zu minimieren, um den Fehler zu minimieren. \\
Aus dieser Erkenntnis folgt der Algorithmus des adaptiven Gitters. Dieser Algorithmus schafft es ohne weitere Kenntnisse über die exakte Funktion ein Gitter zu entwickeln, das an Problemstellen wie zum Beispiel Singularitäten stark verfeinert ist. Mit dieser Adaptivität folgt jedoch auch potenziell mehr Rechenaufwand, da nicht einmal, sondern mehrfach Approximiert wird. bis auf die letzte werden alle Approximationen wieder verworfen. Sie dienten nur dafür die Fehlerindikatoren zu berechnen.\\
Das adaptive Gitter hat im Vergleich zu einem Gitter mit gleichmäßiger Gitterweite, auf denen eine Approximation vergleichbare Fehler liefert, sehr viel weniger Knoten. Da das Approximieren auf einem Gitter mit n-Knoten in $\mathcal{O}(n^2)$ liegt, kann nicht a priori gesagt werden ob das adaptive Gitter Verfahren schneller ist. Dies hängt von der Differenzialgleichung, dem Gebiet und auch von der Wahl des Parameters Theta ab, der entscheidet wie viele der Elemente in der Triangulierung verfeinert werden. Wird Theta klein gewählt, dann werden viele Elemente verfeinert. Der Algorithmus konvergiert in weniger Schritten als für größere Theta. Jedoch enthält die Triangulierung, die für die selbe Fehlerschwelle entsteht, für größeres Theta weniger Knoten. Soll also eine Approximation einmalig berechnet und gespeichert werden, sollte ein möglichst großes Theta gewählt werden, um Apeicherplatz zu sparen. \\
Mit mehr Wissen über die exakte Lösung ist es auch möglich, ein spezielles Gitter für die Approximation zu entwickeln. Zum Beispiel kann gezeigt werden, dass das die Lösung des Poisson Problems an einspringenden Ecke Singularitäten besitzt. So ist es möglich, an den bekannten Problemstellen lokal zu verfeinern. Wie bei einem Gitter mit gleichmäßiger Gitterweite wird hier nur einmal Approximiert. Wie in der Argumentation oben ist kein Verfahren a priori schneller. Im Nachteil zum adaptiven Gitter kann es hier aber sein, dass der Fehler an den nicht speziell verfeinerten Stellen, den Fehler an den Problemstellen durch Überverfeinerung an diesem und sehr große Gitterweite sonst, überwiegt. Außerdem werden Problemstellen, die vorher nicht bekannt sind, nicht speziell behandelt. \\
Im Allgemeinen gilt, wenn die $\eta_{T}$ für die Approximierung auf dem Anfangsgitter ungefähr gleich für alle T sind, also es keine Stellen gibt, an denen der Fehler besonders groß ist, dass das adaptive Gitterverfahren nicht besonders Rechen effizient ist. Im Gegensatz dazu ist das Verfahren besonders effektiv, wenn die Unterschiede zwischen kleinstem und größtem $\eta_{T}$ besonders groß sind. Dies kann Beispielsweise passieren, wenn die exakte Lösung ein Plateau hat. Im Fall der Phasenfeldgleichung, die die Vermischung zweier Substanzen beschreibt, ist das Verfahren mit auf die Gleichung angepassten $\eta_{T}$, sodass die Effizienz- und Verlässlichkeitsabschätzung erfüllt sind, zum Beispiel sehr effektiv, da am Phasenübergang sehr viele, und in den Phasen nur sehr wenige Knoten benötigt werden, um eine gute Approximierung zu erhalten.