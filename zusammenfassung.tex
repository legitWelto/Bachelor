\chapter{Zusammenfassung}
Mit Hilfe des erfahrungsbasierten, berechenbaren Fehlerindikators
\[
\eta_\mathscr{R}(u_h)=\left(\sum_{T\in\mathscr{T}_h} \eta^2_T(u_h)\right)^{1/2}
\]
mit
$
\eta^2_T(u_h) = h^2_T\|f\|^2_{L^2(T)} + \sum_{S\subset\p T} h_S\|\llbracket \nabla u_h \cdot n_S\rrbracket\|^2_{L^2(S)} 
$, ist es möglich den Fehler einer Approximation des Poisson-Problems zu kontrollieren, also den Fehler nach oben und unten zu beschränken. So kann die Approximation, insbesondere lokal, ohne Kenntnis der exakten Funktion bezüglich der Exaktheit beurteilt werden. Es ist bekannt, dass mit der Gitterweite der Fehler sinkt. So folgt für jedes Element $T \in \mathscr{T}_k$ und Elemente $T_i'\in \mathscr{T}_{k+1}$ mit $\overset{.}{\cup} T_i' =T$, da $\eta_T$ den Fehler nach oben beschränkt, dass $\eta_T\geq \sum \eta_{T'_i}$. Somit wird $\eta_\mathscr{R}$ kleiner, wenn einzelne $T$ verfeinert werden. Da $\eta_\mathscr{R}$ den Fehler nach oben beschränkt, reicht es also $\eta_\mathscr{R}$ zu minimieren, um den Fehler zu minimieren. \\
Aus dieser Erkenntnis folgt der Algorithmus des adaptiven Gitters. Dieser Algorithmus schafft es ohne weitere Kenntnisse über die exakte Funktion ein Gitter zu entwickeln, das an Problemstellen wie zum Beispiel Singularitäten stark verfeinert ist. Mit dieser Adaptivität folgt jedoch auch potenziell mehr Rechenaufwand, da nicht einmal, sondern mehrfach approximiert wird. Bis auf die Letzte werden alle Approximationen wieder verworfen. Sie dienten nur dafür die Fehlerindikatoren zu berechnen.\\
Das adaptive Gitter hat im Vergleich zu einem Gitter mit gleichmäßiger Gitterweite, auf denen eine Approximation vergleichbare Fehler liefert, sehr viel weniger Knoten. Da das Approximieren auf einem Gitter mit n-Knoten in $\mathcal{O}(n^2)$ liegt, kann nicht a priori gesagt werden ob das Adaptive-Gitter-Verfahren schneller ist. Dies hängt von der Differenzialgleichung, dem Gebiet und von der Wahl des Parameters Theta ab, der entscheidet wie viele der Elemente in der Triangulierung verfeinert werden. Wird Theta klein gewählt, werden viele Elemente verfeinert. Der Algorithmus konvergiert in weniger Schritten wie für größeres Theta. Jedoch enthält die Triangulierung, die für dieselbe Fehlerschwelle entsteht, für größeres Theta weniger Knoten. Soll also eine Approximation einmalig berechnet und gespeichert werden, sollte ein möglichst großes Theta gewählt werden, um Speicherplatz zu sparen. \\
Mit mehr Wissen über die exakte Lösung ist es auch möglich, ein spezielles Gitter für die Approximation zu entwickeln. Zum Beispiel kann gezeigt werden, dass die Lösung des Poisson-Problems an einspringenden Ecken Singularitäten besitzt. So ist es möglich, an den bekannten Problemstellen lokal zu verfeinern. Wie bei einem Gitter mit gleichmäßiger Gitterweite wird hier nur einmal approximiert. Auch hier ist kein Verfahren a priori schneller, da eine Approximation einer ganzen Reihe an Approximationen gegenübersteht. Im Nachteil zum adaptiven Gitter kann es hier aber sein, dass der Fehler an den nicht speziell verfeinerten Stellen, den Fehler an den Problemstellen überwiegt. Dies ist möglich wenn das Gitter an den speziell verfeinerten Stellen zu fein und auf dem restlichen Gebiet zu grob ist. Außerdem werden Problemstellen, die vorher nicht bekannt sind, nicht speziell behandelt. \\
Im Allgemeinen gilt: \\Wenn die $\eta_{T}$ für die Approximierung auf dem Anfangsgitter ungefähr gleich für alle T sind, es also keine Stellen gibt, an denen der Fehler besonders groß ist, dass das Adaptive-Gitter-Verfahren nicht besonders recheneffizient ist. Im Gegensatz dazu ist das Verfahren effektiv, wenn die Unterschiede zwischen kleinstem und größtem $\eta_{T}$ groß sind. Dies passiert unter anderem, wenn die exakte Lösung ein Plateau hat. Die Phasenfeldgleichung beschreibt die Vermischung zweier Substanzen. Mit auf die Gleichung angepassten $\eta_{T}$, die die Effizienz- und Verlässlichkeitsabschätzung erfüllen, ist das Adaptive-Gitter-Verfahren sehr effektiv. Um eine gute Approximierung zu erhalten, werden am Phasenübergang sehr viele und in den Phasen nur sehr wenige Knoten benötigt. \\
