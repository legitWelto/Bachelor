\chapter{Einleitung}
\renewcommand{\thepage}{\arabic{page}}
\setcounter{page}{1}
Um eine genauere Approximation einer Differentialgleichung zu erhalten, müssen wir die Gitterweite verringern. Doch mit mehr Gitterpunkten steigt auch der Rechenaufwand. Erfahrungsgemäß können wir feststellen, dass der Fehler, also die Differenz zwischen Approximation und exakter Lösung, an manchen Stellen größer ist als an anderen. Um die Rechenzeit effizient zu nutzen, wollen wir das Gitter nur an diesen Stellen verfeinern. Um dies möglich zu machen, ohne die exakte Lösung zu kennen, brauchen wir berechenbare Indikatoren, die den Approximationsfehler kontrollieren. \\
Diese Arbeit stellt zunächst einen solchen Indikator für das Poisson-Problem $-\Delta u =f$ vor und liefert dann  Abschätzungen, die zeigen, dass der Indikator den Fehler kontrolliert. Im Anschluss wird der Algorithmus des adaptiven Gitters vorgestellt, der diesen Fehlerindikator nutzt, um das Approximationsgitter lokal bei großen Fehlerindikatoren zu verfeinern. Dabei muss sichergestellt werden, dass konforme Triangulierungen entstehen. \\
Als Abschluss werden die Ergebnisse auf ein Fallbeispiel angewendet und analysiert, inwiefern der entwickelte Algorithmus die Approximation verändert.