\chapter{Einleitung}
\renewcommand{\thepage}{\arabic{page}}
\setcounter{page}{1}
Um eine genauere Approximation einer Differentialgleichung auf einem uniformen Gitter zu erhalten, muss die Gitterweite verringert werden. Doch mit mehr Gitterpunkten steigt auch der Rechenaufwand. Erfahrungsgemäß kann festgestellt werden, dass der Fehler, also die Differenz zwischen Approximation und exakter Lösung, an manchen Stellen größer ist als an anderen. Um die Rechenzeit effizient zu nutzen, soll das Gitter an diesen Stellen särker verfeinert werden. Um dies möglich zu machen, ohne die exakte Lösung zu kennen, werden berechenbare Indikatoren, die den Approximationsfehler kontrollieren, benötigt. \\
Diese Arbeit stellt zunächst einen Indikator für das Poisson-Problem $-\Delta u =f$ vor und liefert dann  Abschätzungen, die zeigen, dass der Indikator den Fehler kontrolliert. Im Anschluss wird der Algorithmus des adaptiven Gitters vorgestellt. Dieser nutzt den Fehlerindikator, um das Approximationsgitter lokal bei großen Fehlerindikatoren zu verfeinern. Dabei muss sichergestellt werden, dass konforme Triangulierungen entstehen. \\
Zum Schluss werden die Ergebnisse auf ein Fallbeispiel angewendet und analysiert, inwiefern der entwickelte Algorithmus die Approximation verändert.